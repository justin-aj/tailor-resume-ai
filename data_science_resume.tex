%-------------------------
% Resume in Latex
% Author : Jake Gutierrez
% Based off of: https://github.com/sb2nov/resume
% License : MIT
%------------------------

\documentclass[letterpaper,11pt]{article}

\usepackage{latexsym}
\usepackage[empty]{fullpage}
\usepackage{titlesec}
\usepackage{marvosym}
\usepackage[usenames,dvipsnames]{color}
\usepackage{verbatim}
\usepackage{enumitem}
\usepackage[hidelinks]{hyperref}
\usepackage{fancyhdr}
\usepackage[english]{babel}
\usepackage{tabularx}
\usepackage{amsmath} 
\input{glyphtounicode}


%----------FONT OPTIONS----------
% sans-serif
% \usepackage[sfdefault]{FiraSans}
% \usepackage[sfdefault]{roboto}
% \usepackage[sfdefault]{noto-sans}
% \usepackage[default]{sourcesanspro}

% serif
% \usepackage{CormorantGaramond}
% \usepackage{charter}


\pagestyle{fancy}
\fancyhf{} % clear all header and footer fields
\fancyfoot{}
\renewcommand{\headrulewidth}{0pt}
\renewcommand{\footrulewidth}{0pt}

% Adjust margins
\addtolength{\oddsidemargin}{-0.5in}
\addtolength{\evensidemargin}{-0.5in}
\addtolength{\textwidth}{1in}
\addtolength{\topmargin}{-.5in}
\addtolength{\textheight}{1.0in}

\urlstyle{same}

\raggedbottom
\raggedright
\setlength{\footskip}{4.08003pt}

% Sections formatting
\titleformat{\section}{
  \vspace{-4pt}\scshape\raggedright\large
}{}{0em}{}[\color{black}\titlerule \vspace{-5pt}]

% Ensure that generate pdf is machine readable/ATS parsable
\pdfgentounicode=1

%-------------------------
% Custom commands
\newcommand{\resumeItem}[1]{
  \item\small{
    {#1 \vspace{-2pt}}
  }
}

\newcommand{\resumeSubheading}[4]{
  \vspace{-2pt}\item
    \begin{tabular*}{0.97\textwidth}[t]{l@{\extracolsep{\fill}}r}
      \textbf{#1} & #2 \\
      \textit{\small#3} & \textit{\small #4} \\
    \end{tabular*}\vspace{-7pt}
}

\newcommand{\resumeSubSubheading}[2]{
    \item
    \begin{tabular*}{0.97\textwidth}{l@{\extracolsep{\fill}}r}
      \textit{\small#1} & \textit{\small #2} \\
    \end{tabular*}\vspace{-7pt}
}

\newcommand{\resumeProjectHeading}[2]{
    \item
    \begin{tabular*}{0.97\textwidth}{l@{\extracolsep{\fill}}r}
      \small#1 & #2 \\
    \end{tabular*}\vspace{-7pt}
}

\newcommand{\resumeSubItem}[1]{\resumeItem{#1}\vspace{-4pt}}

\renewcommand\labelitemii{$\vcenter{\hbox{\tiny$\bullet$}}$}

\newcommand{\resumeSubHeadingListStart}{\begin{itemize}[leftmargin=0.15in, label={}]}
\newcommand{\resumeSubHeadingListEnd}{\end{itemize}}
\newcommand{\resumeItemListStart}{\begin{itemize}}
\newcommand{\resumeItemListEnd}{\end{itemize}\vspace{-5pt}}

%-------------------------------------------
%%%%%%  RESUME STARTS HERE  %%%%%%%%%%%%%%%%%%%%%%%%%%%%


\hypersetup{
    colorlinks=true,       % Enables colored links
    linkcolor=blue,        % Internal link color
    filecolor=blue,        % File link color
    urlcolor=blue,         % URL link color
    allcolors=blue         % Sets all link elements (optional)
}

\begin{document}

%----------HEADING----------
\begin{center}
    \textbf{\Huge Ajin Frank Justin} \\
    \vspace{1pt}
    857-356-5917 $|$ 
    \href{mailto:ajinfrankj@gmail.com}{\underline{ajinfrankj@gmail.com}} $|$ 
    \href{https://www.linkedin.com/in/ajin-frank-j/}{\underline{linkedin.com/in/ajin-frank-j}} $|$ 
    \href{https://github.com/justin-aj}{\underline{github/justin-aj}} 
    
\end{center}

\vspace{-15pt}  % Adjust this value to reduce the space between the heading and objective

%-----------EDUCATION-----------
\section{Education}
\resumeSubHeadingListStart
\resumeSubheading
{Northeastern University}{\textbf{Sep 2024 - May 2026}}
{\normalfont \textbf{Master of Science in Data Science}}
{\normalfont \textbf{GPA: 4.0/4.0}}
\resumeItemListStart
    \resumeItem{Coursework: Data Mining, Machine Learning, Deep Learning, MLOps, Natural Language Processing}
\resumeItemListEnd

\vspace{-1pt}  % Adjust this value to reduce the space between the heading and objective

\resumeSubheading
{REVA University}{\textbf{Jun 2019 - Jul 2023}}
{\normalfont \textbf{Bachelor of Technology, Computer Engineering}}{\normalfont \textbf{GPA: 9.11/10}}
\resumeSubHeadingListEnd

\vspace{-10pt}  % Adjust this value to reduce the space between the heading and objective

%-----------PROGRAMMING SKILLS-----------
\section{Skills}
 \begin{itemize}[leftmargin=0.05in, label={}]
    \small{\item{
\textbf{ML/AI:} {Regression, Tree-based Ensembles, sklearn, Transformers, LSTM, pytorch, LangChain, Diffusion, LangGraph} \\
    \textbf{Languages:} {Python (Advanced - 5yrs), SQL (Advanced - 3yrs), C\#/C/C++ (Intermediate), GoLang} \\
    \textbf{Data Processing:} {pandas, NumPy, PySpark, MapReduce, Excel, EventHubs, Airflow, Kafka, Snowflake, Databricks, dbt} \\
    \textbf{Cloud/MLOps:} {Azure, AWS, GCP, VertexAI, Docker, Kubernetes, MLFlow, Terraform, GitHub Actions, BitBucket} \\
    \textbf{Visualization/Web Frameworks:} {Matplotlib, PowerBI, Tableau, Django, FastAPI, Flask, ASP.Net, Streamlit} \\
    \textbf{Databases:} {PostgreSQL, MongoDB, MySQL, SparkSQL, Pinecone, OLAP, OLTP, DuckDB, BigQuery}}}
\end{itemize}

\vspace{-15pt}  % Adjust this value to reduce the space between the heading and objective
 
%-----------EXPERIENCE-----------
\section{Experience}
  \resumeSubHeadingListStart

  \resumeSubheading
      {\textcolor{blue}{Data Science Co-op}}{\textbf{Jun 2025 -- Present}}
      {\normalfont \textbf{AARP, Inc.}}
      {Washington DC, USA}
      \resumeItemListStart
      \resumeItem{Developed ML model performance monitoring dashboard in \textbf{Databricks} using \textbf{PySpark, SQL} for scalable data processing, tracking \textbf{10+ performance indicators} across \textbf{25 production models}.}

      \resumeItem{Conducted research of historical model scores across Logistic Regression, Random Forest, Boosting models for \textbf{20+ production use cases}, focusing on residual diagnostics, probability calibration, temporal performance drift.}

    \resumeItemListEnd
    
      \resumeSubheading
      {\textcolor{blue}{DS Research Assistant}}{\textbf{Jan 2025 -- Apr 2025}}
      {\normalfont \textbf{D'Amore-Mckim School of Business, Northeastern University}}
      {Boston, USA}
      \resumeItemListStart
        \resumeItem{Built an NLP ETL pipeline (RegEx, NLTK, spaCy) to preprocess \textbf{2000+ financial filings}, for downstream ML.}
\resumeItem{Created a robust PDF/TXT parser (PyMuPDF, RegEx) to extract entities and structure financial text.}
\resumeItem{Benchmarked LLMs (Gemini, Claude, GPT-4) achieving \textbf{0.86+ F1} in 10-class financial classification.}

    \resumeItemListEnd

\vspace{0pt}

    \resumeSubheading
      {\textcolor{blue}{Data Engineer}}{\textbf{Jun 2023 -- Jun 2024}}
      {\normalfont \textbf{Dynapac, Fayat Group} 
      {\href{https://www.linkedin.com/in/ajin-frank-j/details/recommendations/}{\text{[Manager Recommendations]}}}}{Bangalore, India}
      \resumeItemListStart
        \resumeItem{Restructured Dyn@Lyzer multi-join PostgreSQL telemetry database into partitioned, normalized schema.}
            \resumeItem{Conducted time series forecasting on fuel data using \textbf{ARIMA} models, improving operational \textbf{ROI by 20\%}.}
            \resumeItem{Architected a ETL pipeline orchestrator to handle \textbf{300M+ records from 1000+ nodes}, transform raw data into structured JSON via batch processing with Azure Durable Functions and load to Azure Blob Storage.}
      \resumeItemListEnd
      
% -----------Multiple Positions Heading-----------
%    \resumeSubSubheading
%     {Software Engineer I}{Oct 2014 - Sep 2016}
%     \resumeItemListStart
%        \resumeItem{Apache Beam}
%          {Apache Beam is a unified model for defining both batch and streaming data-parallel processing pipelines}
%     \resumeItemListEnd
%    \resumeSubHeadingListEnd
%-------------------------------------------

  \resumeSubHeadingListEnd

\vspace{-15pt}  % Adjust this value to reduce the space between the heading and objective

%-----------PROJECTS-----------
\section{Projects}
    \resumeSubHeadingListStart
        
        \resumeProjectHeading
          {\textbf{ML Model Performance Monitoring System} $|$ \emph{databricks, pyspark}}{\textbf{Jun 2025 -- Aug 2025}}
          \resumeItemListStart
    \resumeItem{Instrumented statistical drift detection metrics \textbf{(KS, Lift)} to monitor large-scale marketing campaign models.}
\resumeItem{Built a Databricks monitoring framework for \textbf{25 ML models} (LR, GBM, RF) with statistical threshold alerts.}
\resumeItem{Developed automated pyspark pipeline to process \textbf{3B+ scored records} for realtime performance, drift analysis.}
\resumeItem{Identified, flagged \textbf{degrading models} enabling timely retraining, increasing campaign conversion rate by \textbf{15\%}.}



  \resumeItemListEnd
    
    \resumeProjectHeading
          {\textbf{AskNEU RAG System} $|$ \emph{RAG, LangChain, Docker, Pinecone, GCP} \ {\href{https://github.com/justin-aj/AskNEU}{[link]}}}{\textbf{Jan 2025 -- Apr 2025}}
          \resumeItemListStart
            \resumeItem{Architected RAG system with Cohere reranking, direct and Complex Retrieval Framework of documents with query decomposition, context unification using GPT-4.1, Gemini 2.0 APIs and \textbf{LangGraph} for workflow.} 
            \resumeItem{Engineered automated data pipeline to scrape \textbf{50,000+ NEU sites Crawl4AI, Selenium}, chunking, ingest embeddings into \textbf{Pinecone vector DB} with LangChain wrappers to semantic search enabled knowledge base.}
            \resumeItem{Ensured scalability with \textbf{Docker} containers, orchestrating ETL data pipeline(\textbf{Airflow DAGs}), GitHub Actions for CI/CD, leveraging \textbf{Kubernetes} clusters with terraform for automated deployment, \textbf{Grafana} for monitoring.}
          \resumeItemListEnd

          
    % \resumeProjectHeading
    %       {\textbf{AI Banking Assistant} $|$ \emph{transformers, peft, huggingface, pytorch} \ {\href{https://github.com/justin-aj/AIBankingAssistant}{[link]}}}{\textbf{Dec 2024 -- Jan 2025}}
    %       \resumeItemListStart
    %         \resumeItem{Architected question answering and conditional generation tasks, processing \textbf{25000+} QA pairs.}
    %         \resumeItem{Fine-tuned \textbf{T5-small, GPT2-small, DistilBERT} using PEFT(QLoRA) with \textbf{4-bit quantization.}}
    %         \resumeItem{Achieved a \textbf{BLEU score of 0.25} and \textbf{ROUGE-1 F1 score of 0.54} indicating strong model performance in text generation tasks by fine-tuning and benchmarking a \textbf{Falcon-7B}-based model on \textbf{NVIDIA v100 GPU} \textbf{CUDA}.}

          % \resumeItemListEnd
    \resumeProjectHeading
          {\textbf{Food Categorization using Machine Learning} $|$ \emph{sklearn, Random Forest} {\href{https://github.com/justin-aj/USDA-Branded-Foods}{[link]}}}{\textbf{Oct 2024 -- Dec 2024}}
          \resumeItemListStart
            \resumeItem{Developed an automated machine learning system to categorize food products using data from the USDA.}
            \resumeItem{Applied Logistic Regression and Random Forest, achieving \textbf{91.98\% accuracy, 91.87\% F1-Score on 1.7M entries}. Enhanced performance through TF-IDF vectorization, PCA dimensionality reduction, A/B testing.}
            \resumeItem{Derived statistical insights for nutrition, inventory management, leveraging \textbf{hypothesis testing, ANOVA}.}
          \resumeItemListEnd
    \resumeSubHeadingListEnd

%
\end{document}